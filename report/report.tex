% document setup
\documentclass[letterpaper]{article}
\usepackage[utf8]{inputenc}
\usepackage[T1]{fontenc}
\usepackage{textcomp}
\usepackage[spanish]{babel}
\usepackage{amsmath, amssymb}
\usepackage[colorlinks=false]{hyperref}
\usepackage[letterpaper, margin=1in]{geometry}

% figure support
\usepackage{import}
\usepackage{xifthen}
\pdfminorversion=7
\usepackage{pdfpages}
\usepackage{transparent}
\newcommand{\incfig}[1]{%
    \def\svgwidth{\columnwidth}
    \import{./figures/}{#1.pdf_tex}
}
\graphicspath{ {./img/} }
\pdfsuppresswarningpagegroup=1

% document body
\begin{document}
\begin{titlepage}
    \newcommand{\HRule}{\rule{\linewidth}{0.5mm}}
    \center
    \textsc{\LARGE Universidad de Los Lagos}\\[1.5cm]
    \textsc{\Large Inteligencia Artifical}\\[0.5cm]
    \HRule \\[0.4cm]
    { \huge \bfseries Primer Parcial}\\[0.4cm]
    \HRule \\[1.5cm]
    \begin{minipage}{0.4\textwidth}
        \begin{flushleft} \large
            \emph{Autor:}\\
            Diego Muñoz\\
            Cristian Oyarzo\\
            Victor Rodriguez\\
            Sebastian Vidal\\
        \end{flushleft}
    \end{minipage}
    ~
    \begin{minipage}{0.4\textwidth}
        \begin{flushright} \large
            \emph{Profesor:}\\
            Joel Torres\\
        \end{flushright}
    \end{minipage}\\[2cm]
    {\large \today}\\[2cm]
    \includegraphics[width=100px, keepaspectratio]{icinf}\\[1cm]
    \vfill
\end{titlepage}
\tableofcontents
\newpage

\section{Representación}

Consideraremos un arreglo de largo 7 con 1s y 2s con un pivote en el medio representado
por un 0. La condición inicial es la siguiente:

% Estado inicial

La búsqueda termina si el pivote no tiene lugar hábil de movimiento o llega a la
estado final siguiente:

% Estado final

Para el movimiento, el pivote puede intercambiar lugar con celdas contiguas si estas
cumplen con las siguientes condiciones:

\begin{itemize}
    \item Movimientos permitidos:
        \begin{itemize}
            \item l movimiento a la izquierda.
            \item ll dos movimientos a la izquierda.
            \item r movimiento a la derecha.
            \item rr dos movimientos a la derecha.
        \end{itemize}
    \item Ningún movimiento puede dejar que el pivote salga del arreglo.
    \item El pivote puede moverse hacia la izquierda solo si en esa posición hay 1s.
    \item El pivote puede moverse hacia la derecha solo si en esa posición hay 2s.
\end{itemize}

\section{Ejemplos}

% Escribir al menos 5 niveles

\end{document}
